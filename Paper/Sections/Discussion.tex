Currently, the number of wires that our neural network outputs is constant.
However, the complexity of the functions that the wire-fitting function has to model varies from task to task.
To add to the universality of Fido, we plan on implementing a method of dynamically changing the number of wires that the neural network outputs.
Such a method would gauge the variance and bias that the interpolator is experiencing.
Variance is the deviation of a function from its mean.
Bias is the error that results from under fitting.
The wire-fitting function's variance could be measured in a range of actions.
Bias could be measured as the moving average of the wire-fitted interpolator function's error at predicting Q-values, or the moving average of the distance that the wire-fitting function's wires have to move during gradient descent.
Our proposed method would look to reduce $bias^2 + variance$ by increasing the number of wires if there is high bias squared, and decreasing the number of wires if there is high variance.
Each time a wire is removed or added to create a new set of wires, this set of wires would be changed to best model the wire-fitting function formed by the past set of wires.

The topology of our feed-forward neural networks is static throughout Fido's lifetime; the way neurons are arranged in the feed-forward network don't change to fit the task at hand.
To increase the generality of Fido, we would like to research ways to evolve the topology of Fido's neural network as it performs actions and receives reward.
This may mean measuring variance and bias values and determining the direction of $bias^2 + variance$ as outlined above, but may also take the form of an existing variation of back propagation, of which there are many.

Regarding hardware, additional testing on various robot types would be interesting to determine Fido's performance on alternate hardware configurations.
Fido has shown a propensity for mastering advanced control mechanisms, such as balancing on two wheels.
The system could also be applied to robots with changing dynamic models (transforming robots), which must be able to adapt their control mechanisms to suit their current states.
